\documentclass[9pt]{beamer}

% Mathematical functions
% DELETED!
\renewcommand{\Pr}[1]{{\mathbb{P}\left(#1\right) }}
% DELETED!
% DELETED!
% DELETED!
% DELETED!
% DELETED!

% DELETED!
\newcommand{\sufstats}[1]{s\left(#1\right)}
\renewcommand{\exp}[1]{\mbox{exp}\left\{#1\right\}}
\renewcommand{\log}[1]{\mbox{log}\left\{#1\right\}}
\newcommand{\transpose}[1]{{#1}^\mathbf{t}}
\renewcommand{\t}[1]{\transpose{#1}}

\newcommand{\s}[1]{\sufstats{#1}}
\newcommand{\SUFF}{\mathcal{S}}
\newcommand{\Suff}{\mathbf{S}}
\newcommand{\suff}{\mathbf{s}}

\renewcommand{\beta}{\theta}
\newcommand{\weight}{\mathbf{w}}
\newcommand{\Weight}{\mathbf{W}}

% Objects
% DELETED!
% DELETED!
\newcommand{\Graph}{\mathbf{G}}
\newcommand{\graph}{\mathbf{g}}
\newcommand{\GRAPH}{\mathcal{G}}
\newcommand{\Adjmat}{Y}
\newcommand{\adjmat}{y}
\newcommand{\ADJMAT}{\mathcal{Y}}

\newcommand{\INDEPVAR}{\mathcal{X}}
\newcommand{\Indepvar}{X}
\newcommand{\indepvar}{x}

\newcommand{\normconst}{\kappa\left(\params, \Indepvar\right)}

\graphicspath{{./figures/}{.}{./terms/}}


%% NEED THIS FOR CANCY TEX
\usepackage{pstricks}

% Colors
\definecolor{USCCardinal}{HTML}{990000} % 153 0 0 in RGB
\definecolor{USCGold}{HTML}{FFCC00}
\definecolor{USCGray}{HTML}{CCCCCC}

% \bibliography{bibliography.bib}

\def\ergmito{ERGM\textit{ito}}
\def\ergmitos{\ergmito{}\textit{s}}
% Mathematical functions
\newcommand{\isone}[1]{{\boldsymbol{1}\left( #1 \right)}}
\renewcommand{\Pr}[1]{{\mathbb{P}\left(#1\right) }}
\newcommand{\f}[1]{{f\left(#1\right) }}
\newcommand{\Prcond}[2]{{\mathbb{P}\left(#1\vphantom{#2}\;\right|\left.\vphantom{#1}#2\right)}}
\newcommand{\fcond}[2]{{f\left(#1|#2\right) }}
\newcommand{\Expected}[1]{{\mathbb{E}\left\{#1\right\}}}
\newcommand{\ExpectedCond}[2]{{\mathbb{E}\left\{#1\vphantom{#2}\right|\left.\vphantom{#1}#2\right\}}}
\renewcommand{\exp}[1]{\mbox{exp}\left\{#1\right\}}

\newcommand{\Likelihood}[2]{\text{L}\left(#1 \left|\vphantom{#1}#2\right.\right)}

\newcommand{\loglik}[1]{l\left(#1\right)}


% Mathematical Annotation -------------------------------
% Modify this so that it matches the P01 convention overall

% Tree
\newcommand{\phylo}{\Lambda{}} % The actual tree
\newcommand{\aphylo}{D{}}      % The annotated phylogenetic tree
\newcommand{\aphyloObs}{\tilde \aphylo{}} % The observed annotated phylogenetic tree
\newcommand{\parent}[1]{\mathbf{p}\left(#1\right)}
\newcommand{\offspring}[1]{\mathbf{O}\left(#1\right)}
\newcommand{\nodes}{\mathcal{N}{}}
\newcommand{\edges}{\mathcal{E}{}}

\newcommand{\class}[1]{C_{#1}{}}

% Annotations
\newcommand{\Ann}{\mathbf{X}{}} % Matrix of "real" annotations
\newcommand{\ann}[1]{x_{#1}{}} % single element of "real" annotations
\newcommand{\constraints}{\mathcal{C}{}} % Taxon constraints

% Obs Annotations
\newcommand{\AnnObs}{\mathbf{Z}{}}%{Z{}} \mathbf{X}^{obs}{}
\newcommand{\annObs}[1]{z_{#1}{}}%{z{}}  x_{#1}^{obs

% Pred. Annotations
\newcommand{\AnnPred}{\hat X{}}
\newcommand{\annPred}[1]{\hat x_{#1}}

% Leaf nodes
\newcommand{\Leaf}{L{}}

% Shortest path
\newcommand{\Geodesic}{\text{T}{}}
\newcommand{\geodesic}{\tau{}}

\newcommand{\Params}{\Omega{}}
\newcommand{\params}{\omega{}}

% Parameters
\newcommand{\gain}{\mu_{01}{}}
\newcommand{\loss}{\mu_{10}{}}
\newcommand{\misszero}{\psi_{01}{}}
\newcommand{\missone}{\psi_{10}{}}
\newcommand{\proot}{\pi}

\begin{document}
	\frame{
		\frametitle{Likelihood}
		\begin{equation}
			\Prcond{\aphyloObs_n}{x_n} = %
			\sum_{\mathbf{x}^n}\Prcond{\mathbf{x}^n}{x_n}%
			\prod_{i \in \offspring{n}}\Prcond{\aphyloObs_i}{x_i};\label{eq:prob-recursive}
		\end{equation}
	
	\begin{equation}
		\label{eq:transition-probability}\Prcond{\mathbf{x}^n=\mathbf{x}}{\ann{n}} = \frac{\exp{\t{\theta} s(\mathbf{x},\ann{n})}}{\sum_{\mathbf{x}^n}\exp{\t{\theta} s(\mathbf{x}^n, \ann{n})}}
	\end{equation}
	where $\mathbf{x}^n \equiv \{x_i^n\}_{i\in\offspring{n}}$ is an array of size $P$ (functions) $\times$ $|\offspring{n}|$ (offspring) representing the state of node $n$'s offspring, $\ann{n}$ is a binary vector representing the state of node $n$, $\theta$ is a column vector of parameters, and $s(\cdot)$ is a column vector of sufficient statistics which may include terms such as: the total number of functional gains, the number of subfunctionalization or neofunctionalization events, etc
	
	}
	\frame{
		\frametitle{Prediction}
		\begin{equation}
			\notag\Prcond{\mathbf{x}^p = \mathbf{x}}{\aphyloObs}  = %
			 \underbrace{\left\{\prod_{m\in\offspring{p}}\Prcond{\aphyloObs_m}{x_m}\right\}}_{\mbox{Everything below }\mathbf{x}^p} %
			\underbrace{\sum_{x_p}\Prcond{x_p}{\aphyloObs}\frac{%
					\Prcond{\mathbf{x}^p = \mathbf{x}}{x_p}%
				}{%
					\Prcond{\aphyloObs_p}{x_p}
			}}_{\mbox{Everything above }\mathbf{x}^p}
		\end{equation}
	}
\end{document}